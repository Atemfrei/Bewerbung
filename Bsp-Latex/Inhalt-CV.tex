\titel{Lebenslauf}

\section{Persönliche Daten}
\cvitem{Name}										{\nachname}
\cvitem{Vorname}								{\vorname}
\cvitem{Geboren}								{5. September 1981 in Musterstadt}
\cvitem{Nationalität}						{deutsch}
\cvitem{Familienstand}					{ledig}
\cvitem{Anschrift}							{\strasse \ in \PLZ \ \wohnort}
\cvitem{Telefon}								{\mobilNr}
\cvitem{Email}									{\myEmail}


\section{Education}
\cventry{year--year}{Degree}{Institution}{City}{\textit{Grade}}{Description}  % arguments 3 to 6 can be left empty
\cventry{year--year}{Degree}{Institution}{City}{\textit{Grade}}{Description}

\section{Master thesis}
\cvitem{title}{\emph{Title}}
\cvitem{supervisors}{Supervisors}
\cvitem{description}{Short thesis abstract}

\section{Experience}
\subsection{Vocational}
\cventry{year--year}{Job title}{Employer}{City}{}{General description no longer than 1--2 lines.\newline{}%
Detailed achievements:%
\begin{itemize}%
\item Achievement 1;
\item Achievement 2, with sub-achievements:
  \begin{itemize}%
  \item Sub-achievement (a);
  \item Sub-achievement (b), with sub-sub-achievements (don't do this!);
    \begin{itemize}
    \item Sub-sub-achievement i;
    \item Sub-sub-achievement ii;
    \item Sub-sub-achievement iii;
    \end{itemize}
  \item Sub-achievement (c);
  \end{itemize}
\item Achievement 3.
\end{itemize}}
\cventry{year--year}{Job title}{Employer}{City}{}{Description line 1\newline{}Description line 2}
\subsection{Miscellaneous}
\cventry{year--year}{Job title}{Employer}{City}{}{Description}

\section{Languages}
\cvitemwithcomment{Language 1}{Skill level}{Comment}
\cvitemwithcomment{Language 2}{Skill level}{Comment}
\cvitemwithcomment{Language 3}{Skill level}{Comment}

\section{Computer skills}
\cvdoubleitem{category 1}{XXX, YYY, ZZZ}{category 4}{XXX, YYY, ZZZ}
\cvdoubleitem{category 2}{XXX, YYY, ZZZ}{category 5}{XXX, YYY, ZZZ}
\cvdoubleitem{category 3}{XXX, YYY, ZZZ}{category 6}{XXX, YYY, ZZZ}

\section{Interests}
\cvitem{hobby 1}{Description}
\cvitem{hobby 2}{Description}
\cvitem{hobby 3}{Description}

\section{Extra 1}
\cvlistitem{Item 1}
\cvlistitem{Item 2}
\cvlistitem{Item 3. This item is particularly long and therefore normally spans over several lines. Did you notice the indentation when the line wraps?}

\section{Extra 2}
\cvlistdoubleitem{Item 1}{Item 4}
\cvlistdoubleitem{Item 2}{Item 5\cite{book1}}
\cvlistdoubleitem{Item 3}{Item 6. Like item 3 in the single column list before, this item is particularly long to wrap over several lines.}

\section{References}
\begin{cvcolumns}
  \cvcolumn{Category 1}{\begin{itemize}\item Person 1\item Person 2\item Person 3\end{itemize}}
  \cvcolumn{Category 2}{Amongst others:\begin{itemize}\item Person 1, and\item Person 2\end{itemize}(more upon request)}
  \cvcolumn[0.5]{All the rest \& some more}{\textit{That} person, and \textbf{those} also (all available upon request).}
\end{cvcolumns}

% Publications from a BibTeX file without multibib
%  for numerical labels: \renewcommand{\bibliographyitemlabel}{\@biblabel{\arabic{enumiv}}}% CONSIDER MERGING WITH PREAMBLE PART
%  to redefine the heading string ("Publications"): \renewcommand{\refname}{Articles}
\nocite{*}
\bibliographystyle{plain}
\bibliography{publications}                        % 'publications' is the name of a BibTeX file

\vfill
\signature